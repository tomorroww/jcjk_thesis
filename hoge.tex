\documentclass[12pt, a4paper, titlepage]{jarticle}
\usepackage[left = 25truemm, top = 25truemm, bottom = 25truemm, right = 25truemm]{geometry}
\pagestyle{empty}
\usepackage{listings}

%\setlength{\textwidth}{35zw}
%\setlength{\textheight}{30\baselineskip}

\makeatletter
\def\mojiparline#1{
	\newcounter{mpl}
	\setcounter{mpl}{#1}
	\@tempdima=\linewidth
	\advance\@tempdima by-\value{mpl}zw
	\addtocounter{mpl}{-1}
	\divide\@tempdima by \value{mpl}
	\advance\kanjiskip by\@tempdima
	\advance\parindent by\@tempdima
}
\makeatother
\def\linesparpage#1{
	\baselineskip=\textheight
	\divide\baselineskip by #1
	}


\lstset{language=Python}

\title{文字認識精度の向上のために}
\author{tomorrow}
\date{}

\begin{document}
	\maketitle
	% 一行あたり文字数の指定
	\mojiparline{35}
	% 1ページあたり行数の指定
	\linesparpage{30}

	\tableofcontents
	\clearpage

	\section{はじめに}
		\subsection{概要}
			この論文は「文字認識の精度を向上させるためには
			どうすればいいか」という問題に対して,画像の
			処理や変換,機械学習やディープラーニングの手法を
			変えて実験をすることで,文字認識の最適な手法を
			考えるものである.文字認識は文字の中で最も簡単で
			基本的である数字で実験を行う.
	\section{基礎知識}
		\subsection{理論}
			\subsubsection{}
			この論文で用いる文字認識
			の方法を説明する.今回認識する数字の画像は,縦28px
			,横28px,文字の部分が黒,背景が白となっている.
			画像中のそれぞれのピクセルにおける黒色の濃さ
			を表した値をデータとして捉えていく.
			前述のように,今回扱う画像は縦28px,横28px,つまり
			784次元のベクトルとして表現されている.
			\begin{eqnarray}
				\mathrm{batch\_num} : データの個数 \nonumber \\
				\mathrm{x = [batch\_num , 784]} \nonumber
			\end{eqnarray}
			これを用いて,0〜9の中から正しい物を導き出すために
			重みとして,行列[784, 10]をかけることで,
			10通りの結果が生まれ,それにバイアスとして[10]を付け足す
			ことで,10種類の数字に対する出力となる.
			この重みはそれぞれのピクセルの密度が濃い部分がそれぞれの
			数字であることを肯定する証拠ならば正,反する証拠ならば,
			負とするものである.つまり,それぞれのピクセルにおける濃さ
			とそのピクセルが示す数字を表す重みの積の和10種類を求めて
			いる.
			\begin{eqnarray}
				\mathrm{weight} & = & [784 , 10] \nonumber \\
				\mathrm{bias} & = & [10] \nonumber\\
				\mathrm{y} & = & \mathrm{x \times weight + bias} \nonumber
			\end{eqnarray}
			これらの値が大きい物がその数字を表している可能性高いと
			とわかるのでsoftmax関数を用いて確率に変換することで,
			その数字が表している数字が何かを判別することができる.
			\begin{eqnarray}
				\mathrm{p = softmax(y) }\nonumber
			\end{eqnarray}
			これまで述べてきた重みとバイアスを機械学習によって
			求めさせるために,誤差関数としてクロスエントロピーを用いる.
			クロスエントロピーは,実際の答えと現在の重みとバイアスにより
			導かれた答えとの差を現すので,これが最小になるように
			gradient descentアルゴリズムを用いることで,
			重みとバイアスを計算していく.これをくりかえすことで
			最適な重みとバイアスを決定しようとする.
			\begin{eqnarray}
				\mathrm{cross\_entropy = \sum_{i}^{} y'_ilog(p_i)} \nonumber
			\end{eqnarray}
			これによって得られた重みとバイアスを用いて,テスト用の
			データと照らし合わせることで,文字認識の精度を測定する
			ことができる.
			\subsubsection{}

		\subsection{プログラム}
			以上に示した方法によって文字の認識精度を調べるために
			MNISTを学習用データとして,TensorFlowを用いて
			機械学習を行い,テスト用データを使って,認識精度の
			測定を行う.以下にそのソースコードを示す.
			\lstinputlisting[label=src1, caption=Pythonのコード]{main.py}
	\section{問題提起}
		\subsection{問題点}
			2章に示したような方法を用いて文字認識の精度を測ると
			認識精度は平均して91.33%というけっかになった.
			一見すると,精度が高く見えるが,機械学習の中では
			高いというよりもむしろ低い値で,改善の余地がある.
			まず第一に,
		\subsection{問題点の改善点}
			\subsubsection{学習回数}
			\subsubsection{学習アルゴリズム}

	\section{実験方法}
	\section{実験結果}
	\section{結論}

	\begin{thebibliography}{9}
		\bibitem{キー1} hoge
	\end{thebibliography}
\end{document}
